\documentclass[12pt, a4paper]{article}

\usepackage[utf8]{inputenc}
\usepackage[T5]{fontenc}
\usepackage{amsmath, amssymb}
\usepackage{pdfpages}
\usepackage{graphicx}
\usepackage{float}

\newcommand{\R}{\mathbb{R}}

\newcommand{\lecture}[3]{%
  \section*{Lecture #1: #3 (#2)}
}


\begin{document}

\lecture{24}{Mon 20 Oct 2025 18:58}{Theme 24}

\section*{Limit when x $\to $  $\infty$}
\textbf{[Example](page 161)\{this is form $\infty - \infty$\} give :} $\lim\limits_{x \to -\infty}\left( x^3-2x \right)=\lim\limits_{x \to -\infty}x^3\left( 1-\frac{2}{x^2} \right) = -\infty$ because
\begin{cases}
    $\lim\limits_{x \to -\infty}x^3 = -\infty$ \\
    $\lim\limits_{x \to -\infty} = \left( 1-\frac{2}{x^2} \right) = 1 > 0 $ 
\end{cases} \par
$\implies -\infty \cdot 1 = -\infty$ should $\lim\limits_{x \to -\infty}\left( x^3-2x \right)= -\infty$\par
\textbf{Example (page 162)form } $\frac{\infty}{\infty}$\par
\textbf{(1)} $\lim\limits_{x \to +\infty}\frac{2x^2+x-1}{x^2-3x+1} =\lim\limits_{x \to +\infty}\frac{\frac{2x^2}{x^2} + \frac{x}{x^2}-\frac{1}{x^2}}{\frac{x^2}{x^2}-\frac{3x}{x^2}+\frac{1}{x^2}} = \lim\limits_{x \to +\infty}\frac{2+\frac{1}{x}-\frac{1}{x^2}}{1-\frac{3}{x}+\frac{1}{x^2}}= \frac{2 + 0 + 0}{1-0+0}= 2$ \par
\textbf{Method 2:}\par
$\frac{2x^2}{x^2} = 2$ this is quick tip if degree of the numerator = degree of the denominator\par
\textbf{(2)}  $\lim\limits_{x \to -\infty}\frac{-x+1}{2x^2+x-1}= \frac{-\frac{x}{x^2}+\frac{1}{x^2}}{\frac{2x^2}{x^2}+\frac{x}{x^2}-\frac{1}{x^2}} = \frac{\frac{-1}{x}+\frac{1}{x^2}}{2+\frac{1}{x}-\frac{1}{x^2}} = \frac{0+0}{2+0-0}= 0$ \par
\textbf{(3)} $\lim\limits_{x \to +\infty}\frac{x^3+3x^2+2}{2x+1}= \frac{\frac{x^3}{x^3} + \frac{3x^2}{x^3}+\frac{2}{x^3}}{\frac{2x}{x^3}+\frac{1}{x^3}}= \frac{1+\frac{3}{x}+\frac{2}{x^3}}{\frac{2}{x^2}+\frac{1}{x^3}}= \frac{1+0+0}{0+0}=+\infty$ \par
\section*{Summary (Quick Tip)}

You only need to compare the highest powers (degrees) of the numerator and denominator:

\begin{itemize}
    \item \textbf{If the degree of the numerator equals the degree of the denominator:} \\
    The limit equals the ratio of the leading coefficients of the numerator and the denominator. \\
    (As in Example 1)

    \item \textbf{If the degree of the numerator is less than the degree of the denominator:} \\
    The limit equals $0$. \\
    (As in Example 2)

    \item \textbf{If the degree of the numerator is greater than the degree of the denominator:} \\
    The limit equals $\infty$ or $-\infty$. \\
    (As in Example 3)
\end{itemize}
\textbf{Example 1(page 162):}\par
$(1) \lim\limits_{x \to -\infty}\frac{x^3-3x+1}{5-2x}= -\infty;$ \par
$(2) \lim\limits_{x \to -\infty}\frac{1-3x^2-x^3}{4x^2+1}=+\infty;$ \par
$(3) \lim\limits_{x \to +\infty}\frac{3x^2-2x+1}{2x^2-x+5}=\frac{3}{2};$ \par
$(4) \lim\limits_{x \to -\infty}\frac{x+1}{2x^2-x+1}=0;$\par
\textbf{Example 2(page 163)\{Note: $\sqrt{x^2}=|x| $\}:}\par
$\lim\limits_{x \to -\infty}\frac{\sqrt{4x^2-x+1}}{x+1}= \frac{\frac{\sqrt{4x^2-x+1}}{x} }{\frac{x+1}{x}} $ \par
$\implies \frac{\sqrt{4x^2-x+1}}{x} = \frac{\sqrt{x^2\left( \frac{4x^2}{x^2}-\frac{x}{x^2}+\frac{1}{x^2} \right) } }{x}=\frac{|x|\sqrt{\left( 4-\frac{1}{x}+\frac{1}{x^2} \right) } }{x}= \frac{-x\sqrt{\left( 4-\frac{1}{x}+\frac{1}{x^2} \right) } }{x} =-\sqrt{\left( 4-\frac{1}{x}+\frac{1}{x^2} \right)} =-\sqrt{4} = -2$\par
$\implies \frac{x+1}{x} = 1 + \frac{1}{x} = 1$ \par
$\implies \lim\limits_{x \to -\infty}\frac{\sqrt{4x^2-x+1}}{x+1}= \lim\limits_{x \to -\infty}\frac{\frac{\sqrt{4x^2-x+1}}{x} }{\frac{x+1}{x}}  = -\frac{2}{1}= -2$ \par

\textbf{Example 3(page 163):\{Note: $\sqrt{x^2}=|x| $\}}\par
$(1) \lim\limits_{x \to +\infty}\frac{\sqrt{x^2+1}+x }{x+1}= \lim\limits_{x \to +\infty}\frac{\frac{\sqrt{x^2+1}+\frac{x}{x} }{x}}{\frac{x+1}{x}}$ \par
$\implies \frac{\sqrt{x^2+1}+\frac{x}{x}}{x} = \frac{\sqrt{x^2\left( 1+\frac{1}{x^2} \right) }+1}{x}=\frac{|x|\sqrt{\left( 1+\frac{1}{x^2} \right) }+1 }{x}=  \frac{x\sqrt{\left( 1+\frac{1}{x^2} \right) }+1 }{x}= \sqrt{\left( 1+\frac{1}{x^2} \right) }+1=\sqrt{\left( 1+0 \right) }+1 = 2 $ \par
$\implies \frac{x+1}{x}=1 + \frac{1}{x} = 1$ \par
$\implies  \lim\limits_{x \to +\infty}\frac{\sqrt{x^2+1}+x }{x+1}= \lim\limits_{x \to +\infty}\frac{\frac{\sqrt{x^2+1}+\frac{x}{x} }{x}}{\frac{x+1}{x}} = \frac{2}{1} = 2$ \par

$(2) \lim\limits_{x \to -\infty}\frac{\sqrt{x^2-x}-\sqrt{4x^2+1}  }{2x+3}= \frac{\frac{\sqrt{x^2-x} }{x}-\frac{\sqrt{4x^2+1} }{x}}{\frac{2x+3}{x}} $ \par
$\implies \frac{\sqrt{x^2-x} }{x}= \frac{\sqrt{x^2\left( 1-\frac{1}{x} \right) } }{x}= \frac{|x|\sqrt{\left( 1-\frac{1}{x} \right) } }{x}= \frac{-x\sqrt{\left( 1-\frac{1}{x} \right) } }{x}= -\sqrt{\left( 1-\frac{1}{x} \right) }  = -\sqrt{1}  =  -1$ \par
$\implies \frac{-\sqrt{4x^2+1} }{x}= \frac{-\sqrt{x^2\left( 4+\frac{1}{x^2} \right) } }{x}=\frac{-|x|\sqrt{\left( 4+\frac{1}{x^2} \right) } }{x}=\frac{-(-x)\sqrt{\left( 4+\frac{1}{x^2} \right) } }{x}=\frac{x\sqrt{\left( 4+\frac{1}{x^2} \right) } }{x}=\sqrt{\left( 4+\frac{1}{x^2} \right) } =2$ \par
$\frac{2x+3}{x} = 2+\frac{3}{x} = 2$\par
$\implies \lim\limits_{x \to -\infty}\frac{\sqrt{x^2-x}-\sqrt{4x^2+1}  }{2x+3}= \frac{\frac{\sqrt{x^2-x} }{x}-\frac{\sqrt{4x^2+1} }{x}}{\frac{2x+3}{x}} = \frac{-1+2}{2}=\frac{1}{2}$ \par
\textbf{Example 4(page 163)}Give $\lim\limits_{x \to +\infty}\frac{ax^2+bx+3}{cx^3+3x+2}=2;$ know $a,b,c \in \R.$ Caculate $a+b+c?$\par
\textbf{Solve:}\par
Let degree of the numerator = A;\par
Let degree of the denominator = B;\par
We know : \par
Case 1: $A = B \implies$ The limit equals the ratio the leading coefficients of the A and B;\par
Case 2: $A < B \implies $ The limit = 0; \par
Case 3: $A > B \implies$ The limit = $\infty$ or $-\infty$\par
Given : $\lim\limits_{x \to +\infty}\frac{ax^2+bx+3}{cx^3+3x+2} = 2$ only applies to case 1 (A = B)\par 
$\implies c = 0 \implies \lim\limits_{x \to +\infty}\frac{ax^2+bx+3}{3x+2} = 2$ Not yet consistent with the given condition, so let $a = 0$\par
$\implies \lim\limits_{x \to +\infty}\frac{bx+3}{3x+2} = 2 \implies$ satisfy given \par
$\implies  \lim\limits_{x \to +\infty}\frac{bx+3}{3x+2}=2 \implies \frac{b}{3} = 2 \implies b = 6 $ \par
$\implies a + b + c = 6$\par
\textbf{Additional exercises(page: 163)}\par
Find limit:\par
$(1) \lim\limits_{x \to -\infty}\frac{x^2+x+1}{2x^3+2x+5}= 0$(Case 2) \par
$(2) \lim\limits_{x \to +\infty}\frac{2x^2+1}{x^3-3x^2+2}=0$(Case 2) \par
$(3) \lim\limits_{x \to +\infty}\frac{\sqrt[3]{x^2-x+1} }{5x^2-1}=0$(Note: $\sqrt[n]{x^m}=x^{\frac{m}{n}}) $ \par
$(4) \lim\limits_{x \to -\infty}\frac{\sqrt{x^2-x+1} }{2x+1}= \lim\limits_{x \to -\infty}\frac{\frac{\sqrt{x^2-x+1} }{x}}{\frac{2x+1}{x}}$ \par
$\implies \lim\limits_{x \to -\infty}\frac{\sqrt{x^2-x+1} }{x}=\lim\limits_{x \to -\infty}\frac{\sqrt{x^2\left( 1-\frac{1}{x}+\frac{1}{x^2} \right) } }{x} = \lim\limits_{x \to -\infty}\frac{|x|\sqrt{\left( 1-\frac{1}{x}+\frac{1}{x^2} \right) } }{x}= \lim\limits_{x \to -\infty}\frac{-x\sqrt{\left( 1-\frac{1}{x}+ \frac{1}{x^2} \right) } }{x}= \lim\limits_{x \to -\infty}-\sqrt{\left( 1-\frac{1}{x}+\frac{1}{x^2} \right) }= -1 $\par
$\implies \lim\limits_{x \to -\infty}\frac{2x+1}{x}=\lim\limits_{x \to -\infty}\frac{2x}{x}+\frac{1}{x} = \lim\limits_{x \to -\infty}2+\frac{1}{x}=2$\par
$\implies \lim\limits_{x \to -\infty}\frac{\sqrt{x^2-x+1} }{2x+1}= \lim\limits_{x \to -\infty}\frac{\frac{\sqrt{x^2-x+1} }{x}}{\frac{2x+1}{x}}=-\frac{1}{2}$ \par
$(5) \lim\limits_{x \to -\infty}\frac{\sqrt[3]{x^3-1} }{\sqrt{ 2x^2+1}}=\lim\limits_{x \to -\infty}\frac{\frac{\sqrt[3]{x^3-1} }{x}}{\frac{\sqrt{2x^2+1} }{x}}$ \par
$\implies \lim\limits_{x \to -\infty}\frac{\sqrt[3]{x^3-1} }{x} = \lim\limits_{x \to -\infty}\frac{\sqrt[3]{x^3\left( 1-\frac{1}{x^3} \right) } }{x}=\lim\limits_{x \to -\infty}\frac{x\sqrt[3]{\left( 1-\frac{1}{x^3} \right) } }{x}= \lim\limits_{x \to -\infty}\sqrt[3]{\left( 1- \frac{1}{x^3} \right) }=1$\par
$\implies \lim\limits_{x \to -\infty}\frac{\sqrt{2x^2+1} }{x}=\lim\limits_{x \to -\infty}\frac{\sqrt{x^2\left( 2 + \frac{1}{x^2} \right) } }{x}= \lim\limits_{x \to -\infty}\frac{|x|\sqrt{\left( 2+\frac{1}{x^2}\right)} }{x}=\lim\limits_{x \to -\infty}\frac{-x\sqrt{\left( 2+\frac{1}{x^2} \right) } }{x}= \lim\limits_{x \to -\infty}-\sqrt{\left( 2+\frac{1}{x^2} \right) }= -\sqrt{2}  $\par
$\implies \lim\limits_{x \to -\infty}\frac{\sqrt[3]{x^3-1} }{\sqrt{ 2x^2+1}}=\lim\limits_{x \to -\infty}\frac{\frac{\sqrt[3]{x^3-1} }{x}}{\frac{\sqrt{2x^2+1} }{x}}=\frac{1}{-\sqrt{2} }=\frac{-\sqrt{2} }{2} $ \par
$(6) \lim\limits_{x \to -\infty}\frac{\sqrt[3]{x^6+x^4+x^2+1}}{\sqrt{2x^2+1}} = \lim\limits_{x \to -\infty}\frac{\frac{\sqrt[3]{x^6+x^4+x^2+1} }{x^2}}{\frac{\sqrt{2x^2+1} }{x^2}}$ \par
$\implies \lim\limits_{x \to -\infty}\frac{\sqrt[3]{x^6+x^4+x^2+1} }{x^2}=\lim\limits_{x \to -\infty}\frac{\sqrt[3]{x^6+x^4+x^2+1}}{\sqrt[3]{x^6} }= \lim\limits_{x \to -\infty}\sqrt[3]{\frac{x^6+x^4+x^2+1}{x^6}}  =1 $ \par
$\implies \lim\limits_{x \to -\infty}\frac{\sqrt{2x^2+1}}{x^2} =\lim\limits_{x \to -\infty}\frac{\sqrt{2x^2+1} }{\sqrt{x^4}}= \lim\limits_{x \to -\infty}\sqrt{\frac{2x^2+1}{x^4}}=\lim\limits_{x \to -\infty}\sqrt{\frac{2}{x^2}+\frac{1}{x^4}} =0$\par
$\implies \lim\limits_{x \to -\infty}\frac{\sqrt[3]{x^6+x^4+x^2+1}}{\sqrt{2x^2+1}} = \lim\limits_{x \to -\infty}\frac{\frac{\sqrt[3]{x^6+x^4+x^2+1} }{x^2}}{\frac{\sqrt{2x^2+1} }{x^2}}= \frac{1}{0}= +\infty$ \par
\textbf{Method 2:}\par
We can see numerator the greatest is: $\sqrt[3]{x^6} = x^2 $ and all of them are positive.$\implies $ numerator $=+\infty$\par
We can see denominator is $\sqrt{2x^2+1}\implies $ alway positive \par
We know : $x^2 > x^1$(The degree of the numerator is greater than the degree of the denominator)\par
$\implies \frac{positive}{positve}= +\infty$\par
\textbf{Method 3:}\par
The greatest degree of numerator is : $\sqrt[3]{x^6}=x^2 $ \par
The greatest degree of denominator is : $\sqrt{2x^2} = \sqrt{2}\cdot \sqrt{x^2} $(because $x\to -\infty$ should $|x|=-x$) \par
$\implies$ The limit after rewriting is : $\lim\limits_{x \to -\infty}\frac{x^2}{\sqrt{2}(-x)}= \frac{x}{-\sqrt{2}}=\frac{negative}{negative}=+\infty$\par
\textbf{Medthod 4: Using Casio f(x) 580 :>}\par
$(7) \lim\limits_{x \to -\infty}\frac{x-\sqrt{2x^2+1}}{2x+3\sqrt{x^2}+1 }=\lim\limits_{x \to -\infty}\frac{\frac{x-\sqrt{2x^2+1} }{x}}{\frac{2x+3\sqrt{x^2+1} }{x}}$ \par
$\implies \lim\limits_{x \to -\infty}\frac{x-\sqrt{2x^2+1} }{x}= \lim\limits_{x \to -\infty}\frac{-\sqrt{2x^2+1}+x }{x} = \lim\limits_{x \to -\infty}\frac{-\sqrt{x^2\left( 2+\frac{1}{x^2} \right) }+x }{x}= \lim\limits_{x \to -\infty}\frac{-|x|\sqrt{\left( 2+\frac{1}{x^2} \right) }+x }{x}= \lim\limits_{x \to -\infty}\frac{x\sqrt{\left( 2+\frac{1}{x^2} \right) }+x }{x} = \lim\limits_{x \to -\infty}\frac{x\sqrt{\left( 2+\frac{1}{x^2} \right) } }{x}+\frac{x}{x}= \lim\limits_{x \to -\infty}\sqrt{\left( 2+\frac{1}{x^2} \right) }+1 = 1+\sqrt{2} $\par
$\implies \lim\limits_{x \to -\infty}\frac{2x+3\cdot \sqrt{x^2+1} }{x}= \lim\limits_{x \to -\infty}\frac{2x}{x}+\frac{3\cdot \sqrt{x^2+1} }{x}= \lim\limits_{x \to -\infty}2+\frac{3\cdot \sqrt{x^2+1} }{x} = \lim\limits_{x \to -\infty}2+\frac{3\cdot \sqrt{x^2\left( 1+\frac{1}{x^2} \right) } }{x} = \lim\limits_{x \to -\infty}\frac{3\cdot |x|\sqrt{\left( 1+\frac{1}{x^2} \right) } }{x}= \lim\limits_{x \to -\infty}\frac{-3x\cdot \sqrt{\left( 1+\frac{1}{x^2} \right) } }{x}= \lim\limits_{x \to -\infty}2+ -3\cdot \sqrt{\left( 1+\frac{1}{x^2} \right) }=-1$ \par
$\implies \lim\limits_{x \to -\infty}\frac{x-\sqrt{2x^2+1}}{2x+3\sqrt{x^2}+1 }=\lim\limits_{x \to -\infty}\frac{\frac{x-\sqrt{2x^2+1} }{x}}{\frac{2x+3\sqrt{x^2+1} }{x}} = \frac{1+\sqrt{2} }{-1}= -1-\sqrt{2} $\par
$(8) \lim\limits_{x \to +\infty}\sqrt{\frac{x^3+1}{2x^3+5}}=\lim\limits_{x \to +\infty}\sqrt{\frac{\frac{x^3+1}{x^3}}{\frac{2x^3+5}{x^3}}}  = \lim\limits_{x \to +\infty}\sqrt{\frac{1+\frac{1}{x^3}}{2+\frac{5}{x^3}}}=\sqrt{\frac{1+0}{2+0}}=\sqrt{\frac{1}{2}} =\frac{\sqrt{2} }{2}  $ \par
\textbf{Method 2(Case 1)\{degree numerator = degree denominator\}}\par
$\sqrt{\frac{x^3}{2x^3}}=\sqrt{ \frac{1}{2}}= \frac{\sqrt{1} }{\sqrt{2} }= \frac{1}{\sqrt{2} }= \frac{\sqrt{2} }{2}  $\par
\textbf{[Example] (page 164) form $\infty - \infty$}\par
$(1) \lim\limits_{x \to +\infty}\left( \sqrt{x^2+2x}-x \right)=\lim\limits_{x \to +\infty}\frac{\left( \sqrt{x^2+2x} -x \right)\left( \sqrt{x^2+2x} +x \right)}{\left( \sqrt{x^2+2x}+x  \right)}= \lim\limits_{x \to +\infty}\frac{x^2+2x-x}{\left( \sqrt{x^2+2x}+x\right)}=\lim\limits_{x \to +\infty}\frac{2x}{\sqrt{x^2+2x} +x}=\lim\limits_{x \to +\infty}\frac{\frac{2x}{x}}{\frac{\sqrt{x^2\left( 1+\frac{2}{x} \right) } }{x}+\frac{x}{x}}=\lim\limits_{x \to +\infty}\frac{2}{\frac{|x|\sqrt{\left( 1+\frac{2}{x} \right) } }{x}+1}=\lim\limits_{x \to +\infty}\frac{2}{\frac{x\sqrt{\left( 1+\frac{2}{x} \right) } }{x}+1}=\lim\limits_{x \to +\infty}\frac{2}{\sqrt{\left( 1+\frac{2}{x} \right) } +1}= \frac{2}{\sqrt{1+0} +1}= \frac{2}{2}= 1$ \par
$(2) \lim\limits_{x \to +\infty}\left( \sqrt{x+1} -\sqrt{x}  \right)=\lim\limits_{x \to +\infty}\frac{\left( \sqrt{x+1}-\sqrt{x}\right)\left( \sqrt{x+1}+\sqrt{x}\right)}{\left( \sqrt{x+1}+\sqrt{x} \right)}=\lim\limits_{x \to +\infty}\frac{1}{\left( \sqrt{x+1}+\sqrt{x} \right)}=\frac{1}{+\infty}=0 $\par
$(3) \lim\limits_{x \to -\infty}\left( \sqrt{x^2+x+1} +x \right)=\lim\limits_{x \to -\infty}\frac{\left( \sqrt{x^2+x+1}+x \right)\left( \sqrt{x^2+x+1}-x\right)}{\left( \sqrt{x^2+x+1}-x  \right)}=\lim\limits_{x \to -\infty}\frac{x^2+x+1-x^2}{\left( \sqrt{x^2+x+1}-x  \right) }=\lim\limits_{x \to -\infty}\frac{x+1}{\left( \sqrt{x^2+x+1}-x  \right) }=\lim\limits_{x \to -\infty}\frac{\frac{x+1}{x}}{\frac{\sqrt{x^2+x+1}-x }{x}}= \lim\limits_{x \to -\infty}\frac{1+\frac{1}{x}}{\frac{\sqrt{x^2\left( 1+\frac{1}{x}+\frac{1}{x} \right) }-x }{x}}=\lim\limits_{x \to -\infty}\frac{1+\frac{1}{x}}{\frac{|x|\sqrt{\left( 1+\frac{2}{x} \right) }-x }{x}}=\lim\limits_{x \to -\infty}\frac{1+\frac{1}{x}}{\frac{-x\left( \sqrt{1+\frac{2}{x}}\right) -x}{x}}=\lim\limits_{x \to -\infty}\frac{1+\frac{1}{x}}{\frac{-x\sqrt{1+\frac{2}{x}} }{x}-\frac{x}{x}}=\lim\limits_{x \to -\infty}\frac{1+\frac{1}{x}}{-\sqrt{1+\frac{2}{x}}-1}=\frac{1+0}{-\sqrt{1+0} -1}=-\frac{1}{2}$ \par
\textbf{Example 1(page 165):}\par
$(1)\lim\limits_{x \to +\infty}\left( \sqrt{x+3} -\sqrt{x}  \right)=\lim\limits_{x \to +\infty}\frac{\left( \sqrt{x+3}-\sqrt{x}   \right)\left( \sqrt{x+3} +\sqrt{x}  \right)  }{\left( \sqrt{x+3} +\sqrt{x}  \right) }=\lim\limits_{x \to +\infty}\frac{3}{\left( \sqrt{x+3}+\sqrt{x} \right) }= 0 $ \par
$(2) \lim\limits_{x \to +\infty}\left( \frac{4}{\sqrt{x+2}-\sqrt{x-2}  } \right)=\lim\limits_{x \to +\infty}\frac{4\left( \sqrt{x+2}+\sqrt{x-2} \right)  }{\left( \sqrt{x+2}-\sqrt{x-2}   \right)\left( \sqrt{x+2}+\sqrt{x-2}   \right)  }=\lim\limits_{x \to +\infty}\frac{4\left( \sqrt{x+2}+\sqrt{x-2}\right) }{\left( x+2 \right)-(x-2) }= \lim\limits_{x \to +\infty}\left( \sqrt{x+2}+\sqrt{x-2}   \right)=+\infty $ \par
$(3)\lim\limits_{x \to -\infty}\left( \sqrt{x^2-3x}+x  \right)=\lim\limits_{x \to -\infty}\frac{\left( \sqrt{x^2-3x}+x  \right)\left( \sqrt{x^2-3x}-x  \right)  }{\left( \sqrt{x^2-3x}-x\right) }=\lim\limits_{x \to -\infty}\frac{-3x}{\left( \sqrt{x^2-3x}-x  \right) }=\lim\limits_{x \to -\infty}\frac{\frac{-3x}{x}}{\frac{\sqrt{x^2-3x}-x }{x}}=\lim\limits_{x \to -\infty}\frac{-3}{\frac{\sqrt{x^2-3x} }{x}-\frac{x}{x}}=\lim\limits_{x \to -\infty}\frac{-3}{\frac{\sqrt{x^2-3x}}{x}-1}=\lim\limits_{x \to -\infty}\frac{-3}{\frac{\sqrt{x^2\left( 1-\frac{3}{x} \right) } }{x}-1}=\lim\limits_{x \to -\infty}\frac{-3}{\frac{|x|\sqrt{1-\frac{3}{x}} }{x}-1}=\lim\limits_{x \to -\infty}\frac{-3}{\frac{-x\sqrt{1-\frac{3}{x}} }{x}-1}=\lim\limits_{x \to -\infty}\frac{-3}{-\sqrt{1-\frac{3}{x}}-1}=\frac{-3}{-\sqrt{1-0} -1}=\frac{3}{2} $ \par
$(4)\lim\limits_{x \to +\infty}\left( 2x-\sqrt{4x^2+2x-1}\right)=\lim\limits_{x \to +\infty}\frac{\left( 2x-\sqrt{4x^2+2x-1}\right)\left( 2x+\sqrt{4x^2+2x-1}\right) }{\left( 2x+\sqrt{4x^2+2x-1}  \right)}=\lim\limits_{x \to +\infty}\frac{6x^2-2x+1}{\left( 2x+\sqrt{4x^2+2x-1}  \right) }=+\infty$\par
\textbf{Example 2(page 165):Value of $A = \lim\limits_{x \to -\infty}\left( \sqrt{x^2+4x+5}+x  \right)$ =?}\par
$A=\lim\limits_{x \to -\infty}\left( \sqrt{x^2+4x+5}+x\right)=\lim\limits_{x \to -\infty}\frac{\left( \sqrt{x^2+4x+5}+x  \right)\left( \sqrt{x^2+4x+5}-x \right)  }{\left( \sqrt{x^2+4x+5}-x  \right) }=\lim\limits_{x \to -\infty}\frac{4x+5}{\left( \sqrt{x^2+4x+5}-x  \right) }=\lim\limits_{x \to -\infty}\frac{\frac{4x+5}{x}}{\frac{\left( \sqrt{x^2+4x+5}-x  \right) }{x}}=\lim\limits_{x \to -\infty}\frac{\frac{4x}{x}+\frac{5}{x}}{\frac{\sqrt{x^2+4x+5} }{x}-\frac{x}{x} }=\lim\limits_{x \to -\infty}\frac{4+\frac{5}{x}}{\frac{\sqrt{x^2+4x+5}}{x}-1}=\lim\limits_{x \to -\infty}\frac{4+\frac{5}{x}}{\frac{\sqrt{x^2\left( 1+\frac{4}{x}+\frac{5}{x^2}\right)}}{x}-1}=\lim\limits_{x \to -\infty}\frac{4+\frac{5}{x}}{\frac{|x|\sqrt{\left( 1+\frac{4}{x}+\frac{5}{x^2}\right) } }{x}-1}=\lim\limits_{x \to -\infty}\frac{4+\frac{5}{x}}{\frac{-x\sqrt{\left( 1+\frac{4}{x}+\frac{5}{x^2} \right) } }{x}-1}=\lim\limits_{x \to -\infty}\frac{4+\frac{5}{x}}{-\sqrt{1+\frac{4}{x}+\frac{5}{x^2}} -1}=\frac{4}{-2}=-2 $ \par
$\implies A = -2$\par
\textbf{Example 3(page 165):}\par
Give $\lim\limits_{x \to +\infty}\left( \sqrt{x^2+ax}-x  \right)=8.$ Find $a, a \in \R$ \par
$\lim\limits_{x \to +\infty}\left( \sqrt{x^2+ax} -x \right) =\lim\limits_{x \to +\infty}\frac{\left( \sqrt{x^2+ax}-x  \right)\left( \sqrt{x^2+ax}+x  \right)  }{\left( \sqrt{x^2+ax}+x  \right) }=\lim\limits_{x \to +\infty}\frac{ax}{\left(\sqrt{x^2+ax} +x  \right) }=\lim\limits_{x \to +\infty}\frac{\frac{ax}{x}}{\frac{\sqrt{x^2+ax}+x }{x}}=\lim\limits_{x \to +\infty}\frac{a}{\frac{\sqrt{x^2+ax} }{x}+\frac{x}{x}}=\lim\limits_{x \to +\infty}\frac{a}{\frac{\sqrt{x^2\left(1+\frac{a}{x}\right) } }{x}+1}=\lim\limits_{x \to +\infty}\frac{a}{\frac{|x|\sqrt{\left( 1+\frac{a}{x} \right) } }{x}+1}=\lim\limits_{x \to +\infty}\frac{a}{\frac{x\sqrt{\left( 1+\frac{a}{x} \right) } }{x}+1}=\lim\limits_{x \to +\infty}\frac{a}{\sqrt{1+\frac{a}{x}} +1}=\frac{a}{\sqrt{1+0}+1 }=\frac{a}{2}=8 \implies a = 16$\par
\textbf{Additional Exercises(page 165):}\par
$(1)\lim\limits_{x \to -\infty}\left( \sqrt{x^2+4x+1}+x  \right)= \lim\limits_{x \to -\infty}\frac{\left( \sqrt{x^2+4x+1} +x \right)\left( \sqrt{x^2+4x+1} -x \right)  }{\left( \sqrt{x^2+4x+1}-x  \right) }=\lim\limits_{x \to -\infty}\frac{4x+1}{\sqrt{x^2+4x+1}-x}=\lim\limits_{x \to -\infty}\frac{\frac{4x+1}{x}}{\frac{\sqrt{x^2+4x+1} -x}{x}}=\lim\limits_{x \to -\infty}\frac{4+\frac{1}{x}}{\frac{\sqrt{x^2+4x+1} }{x}-\frac{x}{x}}=\lim\limits_{x \to -\infty}\frac{4+\frac{1}{x}}{\frac{\sqrt{x^2+4x+1} }{x}-1}=\lim\limits_{x \to -\infty}\frac{4+\frac{1}{x}}{\frac{\sqrt{x^2\left( 1+\frac{4}{x}+\frac{1}{x^2} \right) } }{x}-1}=\lim\limits_{x \to -\infty}\frac{4+\frac{1}{x}}{\frac{|x|\sqrt{\left( 1+\frac{4}{x}+\frac{1}{x^2} \right)}}{x}-1}=\lim\limits_{x \to -\infty}\frac{4+\frac{1}{x}}{\frac{-x\sqrt{\left( 1+\frac{4}{x}+\frac{1}{x^2} \right) } }{x}-1}=\lim\limits_{x \to -\infty}\frac{4+\frac{1}{x}}{-\sqrt{\left( 1+\frac{4}{x}+\frac{1}{x^2} \right) } -1}=\frac{4+0}{-\sqrt{1+0+0} -1}=\frac{4}{-2}= -2$\par
(2) \lim_{x \to \infty} (\sqrt{x^2 - 4x + 2} - x) &= \lim_{x \to \infty} \frac{(\sqrt{x^2 - 4x + 2} - x)(\sqrt{x^2 - 4x + 2} + x)}{\sqrt{x^2 - 4x + 2} + x} = \lim_{x \to \infty} \frac{-4x + 2}{\sqrt{x^2 - 4x + 2} + x} \\
&= \lim_{x \to \infty} \frac{\frac{-4x + 2}{x}}{\frac{\sqrt{x^2 - 4x + 2} + x}{x}} = \lim_{x \to \infty} \frac{-4 + \frac{2}{x}}{\frac{\sqrt{x^2 - 4x + 2}}{x} + 1} = \lim_{x \to \infty} \frac{-4 + \frac{2}{x}}{\frac{\sqrt{x^2(1 - \frac{4}{x} + \frac{2}{x^2})}}{x} + 1} = \lim_{x \to \infty} \frac{-4 + \frac{2}{x}}{\frac{|x|\sqrt{1 - \frac{4}{x} + \frac{2}{x^2}}}{x} + 1} \\
&= \lim_{x \to \infty} \frac{-4 + \frac{2}{x}}{\frac{x\sqrt{1 - \frac{4}{x} + \frac{2}{x^2}}}{x} + 1} = \lim_{x \to \infty} \frac{-4 + \frac{2}{x}}{\sqrt{1 - \frac{4}{x} + \frac{2}{x^2}} + 1} = \frac{-4 + 0}{\sqrt{1 - 0 + 0} + 1} = -2\par
(3) $\lim_{x \to -\infty} (\sqrt{x^2 + x} - x) = +\infty - (-\infty) = +\infty + \infty = +\infty \text{ \{be careful it not form } \infty - \infty \text{\}}$\par
(4) \lim_{x \to -\infty} (\sqrt{4x^2 + x} + 2x) &= \lim_{x \to -\infty} \frac{(\sqrt{4x^2 + x} + 2x)(\sqrt{4x^2 + x} - 2x)}{\sqrt{4x^2 + x} - 2x} = \lim_{x \to -\infty} \frac{x}{\sqrt{4x^2 + x} - 2x} \\
&= \lim_{x \to -\infty} \frac{\frac{x}{x}}{\frac{\sqrt{4x^2 + x} - 2x}{x}} = \lim_{x \to -\infty} \frac{1}{\frac{\sqrt{4x^2 + x}}{x} - \frac{2x}{x}} = \lim_{x \to -\infty} \frac{1}{\frac{\sqrt{x^2(4 + \frac{1}{x})}}{x} - 2} \\
&= \lim_{x \to -\infty} \frac{1}{\frac{|x|\sqrt{4 + \frac{1}{x}}}{x} - 2} = \lim_{x \to -\infty} \frac{1}{\frac{-x\sqrt{4 + \frac{1}{x}}}{x} - 2} = \lim_{x \to -\infty} \frac{1}{-\sqrt{4 + \frac{1}{x}} - 2} \\
&= \frac{1}{-\sqrt{4 + 0} - 2} = -\frac{1}{4}\par
(5) $\lim_{x \to -\infty} (\sqrt{x^2+2x} - \sqrt{x^2-1}) = \lim_{x \to -\infty} \frac{(\sqrt{x^2+2x} - \sqrt{x^2-1})(\sqrt{x^2+2x} + \sqrt{x^2-1})}{\sqrt{x^2+2x} + \sqrt{x^2-1}} = \lim_{x \to -\infty} \frac{(x^2+2x) - (x^2-1)}{\sqrt{x^2+2x} + \sqrt{x^2-1}} = \lim_{x \to -\infty} \frac{2x+1}{\sqrt{x^2+2x} + \sqrt{x^2-1}} = \lim_{x \to -\infty} \frac{x(2 + \frac{1}{x})}{\sqrt{x^2(1+\frac{2}{x})} + \sqrt{x^2(1-\frac{1}{x^2})}} = \lim_{x \to -\infty} \frac{x(2 + \frac{1}{x})}{|x|\sqrt{1+\frac{2}{x}} + |x|\sqrt{1-\frac{1}{x^2}}} = \lim_{x \to -\infty} \frac{x(2 + \frac{1}{x})}{-x \left( \sqrt{1+\frac{2}{x}} + \sqrt{1-\frac{1}{x^2}} \right)} = \lim_{x \to -\infty} \frac{2 + \frac{1}{x}}{-\left( \sqrt{1+\frac{2}{x}} + \sqrt{1-\frac{1}{x^2}} \right)} = \frac{2+0}{-( \sqrt{1+0} + \sqrt{1-0} )} = \frac{2}{-(1+1)} = \frac{2}{-2} = -1$\par
$(6) \lim\limits_{x \to -\infty}\frac{\sqrt{4x^2+x-2}+x }{3x-1}=\lim\limits_{x \to -\infty}\frac{\left( \sqrt{4x^2+x-2}+x  \right)\left( \sqrt{4x^2+x-2}-x  \right)  }{\left( 3x-1 \right)\left( \sqrt{4x^2+x-2}-x  \right)  }=\lim\limits_{x \to -\infty}\frac{3x^2+x-2}{\left( 3x-1 \right)\left( \sqrt{4x^2+x-2}-x  \right)  }=\lim\limits_{x \to -\infty}\frac{\frac{3x^2+x-2}{x^2}}{\frac{\left( 3x-1 \right)\left( \sqrt{4x^2+x-2}-x  \right)  }{x^2}}=\lim\limits_{x \to -\infty}\frac{3+\frac{1}{x}-\frac{2}{x^2}}{\frac{\left( 3x-1 \right)\left( \sqrt{4x^2+x-2}  \right)  }{x^2}- \frac{x}{x^2}}=\lim\limits_{x \to -\infty}\frac{3+\frac{1}{x}-\frac{2}{x^2}}{\frac{\left( 3x-1 \right)\left( \sqrt{4x^2+x-2}  \right)  }{x^2}-\frac{1}{x}}=\lim\limits_{x \to -\infty}\frac{3+\frac{1}{x}-\frac{2}{x^2}}{\frac{\left( 3x-1 \right)\left( \sqrt{x^2\left( 4+\frac{1}{x}-\frac{2}{x^2} \right) }  \right)  }{x^2}-\frac{1}{x}}=\lim\limits_{x \to -\infty}\frac{3+\frac{1}{x}-\frac{2}{x^2}}{\frac{\left( 3x-1 \right)|x|\sqrt{\left( 4+\frac{1}{x}-\frac{2}{x^2} \right) }  }{x^2}-\frac{1}{x}}=\lim\limits_{x \to -\infty}\frac{3+\frac{1}{x}-\frac{2}{x^2}}{\frac{\left( 3x-1 \right)\left( -x\sqrt{\left( 4+\frac{1}{x}-\frac{2}{x^2} \right) }  \right)  }{x^2}-\frac{1}{x}}=\lim\limits_{x \to -\infty}\frac{3+\frac{1}{x}-\frac{2}{x^2}}{\frac{-3x^2\sqrt{\left( 4+\frac{1}{x}-\frac{2}{x^2} \right) } }{x^2}-\frac{1}{x}}=\lim\limits_{x \to -\infty}\frac{3+\frac{1}{x}-\frac{2}{x^2}}{-3\sqrt{\left( 4+\frac{1}{x}-\frac{2}{x^2} \right) }-\frac{1}{x} }=\frac{3+0-0}{-3\sqrt{\left( 4+0-0 \right) }-0 }=-\frac{1}{2}$\par
$(7)\lim\limits_{x \to +\infty}\left\{\frac{1}{x+1}\cdot \left( \sqrt{x^2+2x}-x \right)\right\}=\lim\limits_{x \to +\infty}\frac{1}{x+1}\cdot \left( \frac{\sqrt{x^2+2x}-x\left( \sqrt{x^2+2x}+x  \right)  }{\left( \sqrt{x^2+2x}+x  \right) } \right)=\lim\limits_{x \to +\infty}\frac{1}{x+1}\cdot \left( \frac{2x}{\sqrt{x^2+2x}+x } \right)=\lim\limits_{x \to +\infty}\frac{2x}{(x+1)\sqrt{x^2+2x}+x }=0$(numerator degree < denominator degree)\par
$(8)\lim\limits_{x \to +\infty}\left( \sqrt{x^2+3x}-\sqrt{x^2-3x}\right)=\lim\limits_{x \to +\infty}\frac{\left( \sqrt{x^2+3x}-\sqrt{x^2-3x}\right)\left( \sqrt{x^2+3x}+\sqrt{x^2-3x}\right)}{\left( \sqrt{x^2+3x}+\sqrt{x^2-3x} \right) }=\lim\limits_{x \to +\infty}\frac{6x}{\left( \sqrt{x^2+3x}+\sqrt{x^2-3x}\right) }=\lim\limits_{x \to +\infty}\frac{\frac{6x}{x}}{\frac{\sqrt{x^2+3x} }{x}+\frac{\sqrt{x^2-3x} }{x}}=\lim\limits_{x \to +\infty}\frac{6}{\frac{\sqrt{x^2\left( 1+\frac{3}{x} \right) } }{x}+\frac{\sqrt{x^2\left( 1-\frac{3}{x} \right) } }{x}}=\lim\limits_{x \to +\infty}\frac{6}{\frac{|x|\sqrt{\left( 1+\frac{3}{x} \right) } }{x}+\frac{|x|\sqrt{\left( 1-\frac{3}{x} \right) } }{x}}=\lim\limits_{x \to +\infty}\frac{6}{\frac{x\sqrt{\left( 1+\frac{3}{x} \right) } }{x}+\frac{x\sqrt{\left( 1-\frac{3}{x} \right) } }{x}}=\lim\limits_{x \to +\infty}\frac{6}{\sqrt{\left( 1+\frac{3}{x} \right) } +\sqrt{\left( 1-\frac{3}{x} \right) } }=\frac{6}{\sqrt{1+0}+\sqrt{1-0}}=3$\par

\end{document}
