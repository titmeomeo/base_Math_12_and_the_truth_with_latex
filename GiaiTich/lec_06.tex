\documentclass[12pt, a4paper]{article}
\usepackage{float}
\usepackage{pdfpages}
\usepackage{graphicx}
\usepackage{amsmath, amssymb}
\newcommand{\R}{\mathbb{R}}
\usepackage[utf8]{inputenc}
\usepackage[T5]{fontenc} % Hỗ trợ tiếng Việt
\usepackage{amsmath}
\usepackage{amssymb}
\usepackage[a4paper, margin=1in]{geometry}

% --- CÁC GÓI BẮT BUỘC CHO VIỆC CHÈN HÌNH (PHẦN BỊ THIẾU) ---
\usepackage{graphicx}      % Gói này anh đã có
\usepackage{import}
\usepackage{pdfpages}
\usepackage{transparent}

% --- ĐỊNH NGHĨA LỆNH \incfig (PHẦN BỊ THIẾU) ---
\newcommand{\incfig}[1]{%
    \def\svgwidth{\columnwidth}
    \import{./figures/}{#1.pdf_tex}
}

% --- ĐỊNH NGHĨA LỆNH \lecture (Phần này của anh đã đúng) ---
\usepackage{xifthen}

\makeatletter
\def\@lecture{}%
\newcommand{\lecture}[3]{
    \ifthenelse{\isempty{#3}}{%
        \def\@lecture{Lecture #1}%
    }{%
        \def\@lecture{Lecture #1: #3}%
    }%
    \subsection*{\@lecture}
    \marginpar{\small\textsf{\mbox{#2}}}
}
\makeatother
% --- KẾT THÚC ĐỊNH NGHĨA LỆNH \lecture ---

\providecommand{\lecture}[6]{%
  \section*{Lecture #1: #3 (#2)}
}

\begin{document}

\lecture{6}{Mon 22 Sep 2025 20:00}{Lecture 6}
\textbf{Lecture 6 : the maximum and minium values of a function (part 2)}\\
\textbf{I.Theory}\\
\textbf{1. Find max, min of function on segment}\\
Suppose the funtion $f(x)$ is continuous on the interval  $[a, b]$ and differentiable on the open interval  $(a, b)$. Then, the rule for finding the maximum and minium values of $f(x)$ on $[a;b]$ is as follows:\\
\textbf{Step 1:} Find the  points $x_1, x_2, x_n$ in the interval $(a, b)$ where the derivaive of the function is zero or does not exist (in simple terms, solve the equation $f'(x) = 0$ ).\\
\textbf{Step 2:} Calculate $f(x_1), f(x_2),\ldots,f(x_n), f(a), f(b)$ (here $f(a)$ and $f(b)$ are values at the endpoints of the interval).\\
\textbf{Step 3:} Compare all the values obtained in the step 2.\\
The largest of these values is maximum value of $f(x)$ on  $[a,b]$ and the smallest is the minimum value of  $f(x)$ on  $[a,b]$\\
\textbf{Example 1.}\\
a)\\
$f(x)$ continuos on interval  $[-2;2]$\\
$f'(x) = 3x^2 - 6x +9x+10$\\
 $f'(x) = 0 \iff 3x^2 - 6x + 9x + 10 = 0 \iff$
\begin{math}
	\left[ \begin{array}{l}
			x = 3 \neq [-2;2] \implies (not-pick)\\
			x = -1 \in [-2;2] \implies (pick) 
	\end{array} \right. 
\end{math}\\v
$f(-1) = 15 \implies \max$\\
$f(-2) = 8$\\
$f(2) = -12 \implies \min$\\
$ \implies$ $\max\limits_{[-2;2]} = 15$ at  $x = 1$\\
$\implies \min\limits_{[-2;2]} = -12$ at $x = 2$\\
b)\\
$f(x)$ continuos on interval  $[1;e^2]$\\
$f'(x) = \frac{(\ln x)' \cdot x - x'\cdot \ln x }{x^2} = \frac{1-\ln x}{x^2}$ \\
$f'(x)=0 \implies 1-\ln x = 0 \iff 1=\ln x \iff x = e \in  [1;e^2]$\\
$f(1)=0$ at  $x=1$\\
$f(e) = \frac{\ln(e)}{e}=\frac{1}{e}$\\
$f(e^2)=\frac{2}{e^2}$ \\
$\min\limits_{[1;e^2]}f(x) = 0$ at $x = 1$\\ 
$\max\limits_{[1;e^2]}f(x) = \frac{1}{e}$ at $x=e$\\
Note:\\
 $(\ln(x))' = \frac{1}{x}$ \\
 $\ln(e)=1$
 $\ln(x) = y \iff e^y = x$\\
 $\ln(x)=1 \iff e^1 = x \iff x = e$\\
c)\\
$f(x)$ continuos on interval  $[0;\ln 2]$\\
$f'(x)=e^x+xe^x \iff e^x(x+1)$\\
$f'(x)=0 \iff e^x(x+1) = 0 \iff 1+x=0 \iff x = -1 \not\in [0;\ln 2]$\\
$f(0) = 0$\\
$f(\ln 2) = \ln 2 \cdot e^\ln(2)$ $=2 \cdot \ln 2$\\
$\implies \min\limits_{[0;\ln 2]}f(x)=0$ at $x=0$\\
$\implies \max\limits_{[0;\ln 2]}f(x)$ at $x=\ln 2$\\
\textbf{Note: $e^\ln 2$ $=2$}\\
\textbf{II.Apply}\\
\textbf{Example 2.}\\
\textbf{Note 1: Find $\max v(t), t \in [0;6]$\\
  Note 2: $v(t) = s'(t)$
} \\
According to the physical meaning of the derivative, we have the instantaneous velocity of a particle:\\
$v(t) = s'(t) = -3t^2 + 8t^2 + 2t + 2$\\
Consider  $v(t)$ on  $[0;6]$; we get  $v(t)$ continuous on the interval  $[0;6]$\\
 $v'(t) = -6t+16$\\
 $v'(t) = 0 \iff t = \frac{8}{3} \in [0;6]$\\
 $v(0) = 2;v(6) = -10, v(\frac{8}{3}) = \frac{70}{3}$ \\
 $\implies \max\limits_{[0;6]}v(t) = \frac{70}{3}$ (m/s) at $t = \frac{8}{3}$ (s)\\
 \textbf{Example 3.}\\
 \textbf{Note:} find $\max\limits_{[0;30]} F(x)$\\
 \textbf{Note:} find x \\
 $F'(x) = 0,05x(30-x)-\frac{1}{40}x^2$ \\
$=\frac{3}{2}x-0,05x^2 - \frac{1}{40}x^2$\\
$= \frac{3}{2}x - \frac{3}{40}x^2$ \\
$F'(x)=0 \iff $
 \begin{math}
 	\left[ \begin{array}{l}
			x=20 \in [0;30]\\
			x=0 \in [0;30]
 	\end{array} \right. 
 \end{math}\\
$F(0) = 0$ at $x=0$\\
$F(30)=0$ at  $x=30$\\
$F(20)=100$ at  $x=20$\\
$\implies \max\limits_{[0;30]}F(x) = 100$ at $x=20$(miligam)\\
\textbf{Example 4.}\\
\textbf{Sumarize :}\\
- if 35 - 1 = 34 $\implies$ then increase by 50 motorbike\\
- if 35 + 1 = 36 $\implies$ decrease by 50 motorbike\\
- know cost price = 30 million / one\\
- bonus distcount  8\% in cost price\\
\textbf{Solve}\\
+) Profit = (revenue - cost)\\
+) revenue = price * quantity sold |Note: if + 1 $\implies -50 \implies +x\implies-50x$\\
+) price = $(35+x) \implies$ $x<0$ price increase,  $x>0$ price decrease\\
+) Quantity sold =  $(1000-50x)$\\
+) profit of one motorbike:  $(35+x)-30+30 \cdot 8\% = 7,4 + x$\\
+) Let sum profit is $P(x) \implies P(x) = (7,4 + x)\cdot(1000-50x)$ \textbf{Note:This is sum profit = profit * quantity sold}\\
$=7400-370x+1000x-50x^2$\\
 $=-50x^2+630x-7400$\\
  $P'(x)=-1000x+630$\\
  $\implies P(x)=0 \iff x=6,3 \implies x > 0 \implies$ \textbf{price decrease}\\
variation table:\\

\begin{figure}[H]
    \centering
    \incfig{figure-4-of-example-4}
    \caption{figure 4 of example 4}
    \label{fig:figure-4-of-example-4}
\end{figure}
$\implies$ Price $=35+6,3=41,3$ million  $\implies D$ \\
\textbf{Example 5.}\\
\textbf{Note: s = v * t(general formula), t = s / v, find x -> find equation timer}\\
\textbf{Sumarize:}\\
+) AB = 4 km\\
+) AM = v = 6km/h(1)\\
+) v of MC = 10km/h(2)\\
+) (1) and (2) is v\\
\textbf{Solve:}\\
$AC = AM + MC \to ( 0 \le x \le 7$ beacause: if MC = MC $\implies$ x = x = 0, else $MC = MB \implies 7 - x \implies x = 7$\\
Let $MC = x \to $ $S_1$(distane)\\
$\iff AC = AM + x$\\
$AM^2=AB^2+BM^2$ Note :  $BM = BC-MC(MC = x \iff BM = 7 - x$\\
$AM^2=4^2+(7-x)^2 \iff AM = \sqrt{16+(7-x)^2} \to S_2$(distance)\\
We get equation timer:\\
$AM = \frac{\sqrt{16+(7-x)^2} }{6},MC=\frac{x}{10}$\\
$AC = AM+MC=\frac{\sqrt{16+(7-x)^2} }{6} + \frac{x}{10}$\\
 Let $AC=T(x) \implies T(x) = \frac{\sqrt{16+(7-x)^2} }{6} + \frac{x}{10}$ Note: $(A-B)^2=A^2-2AB+B^2$\\
 \textbf{Consider:} $T(x) = \frac{\sqrt{x^2-14x+65} }{6} +  \frac{x}{10}$ \\
 $T'(x)=\frac{2x-14}{12\sqrt{x^2-14x+65} } + \frac{1}{10}$ |Note: LCM(12,10) = 60\\
 $=\frac{10x-70}{60\sqrt{x^2-14x+65} } + \frac{6\sqrt{x^2-14x+65} }{60\sqrt{x^2-14x+65} }$\\
 $=\frac{10x-70+6\sqrt{x^2-14x+65} }{60\sqrt{x^2-14x+65}}$ \\
 $T'(x)=0 \iff 10x-70+6\sqrt{x^2-14x+65}=0$ \\
 $\iff 5x-35+3\sqrt{x^2-14x+65}=0 $ \\
 $\iff 5x-35=-3\sqrt{x^2-14x+65}$ \\
$\iff (5x-35)^2=9(x^2-14x+65)$\\
$25x^2-350x+1225=9x^2-126x+585$\\
 $16x^2-224x+640=0 \iff$
 \begin{math}

   \left[ \begin{array}{l}
		   x = 10 \not\in [0;7]\\
		   x = 4  \in [0;7]
   \end{array} \right.  	
 \end{math}
 \begin{math}
 	T(0) = \frac{\sqrt{65} }{6}\\
	T(4) = \frac{37}{30}\\
	T(7) = \frac{41}{30}\\
	\implies \min\limits_{[0;7]} T(x) = \frac{37}{30} \text {,at,} x = 4
 \end{math} \\
 \textbf{III.Homework}\\
 \textbf{Sentence 1.} $\mathbb{D} = \R \implies f(x)$ continuous on the interval $[-1;2]$\\
  $f'(x) = -4x^3 + 24x^2$\\
   $f'(x) = 0 \iff $ 
   \begin{math}
   	\left[ \begin{array}{l}
			x = 6 \not\in [-1;2]\\
			x = 0 \in [-1;2]
   	\end{array} \right. 
   \end{math}\\
$f(-1) = 12$\\ 
 $f(2) = 33$\\
 $f(0) = 1$\\
 $\implies \max\limits_{[-1;2]}f(x) = 33$ at $x = 2$\\
 \textbf{ Sentence2.} $\mathbb{D} = \R \implies f(x)$ continuous on the interval $[2;19]$\\ 
 $f'(X) = 3x^2 -24$\\
  $f'(x) = 0 \iff$
  \begin{math}
  	\left[ \begin{array}{l}
			x = 2\sqrt{2} \in [2;19]\\
			x = -2\sqrt{2} \not\in [2;19]
  	\end{array} \right. 
  \end{math}\\
$f(2) = -40$\\
 $f(19)=6403$\\
 $f(2\sqrt{2})=-32\sqrt{2}$ \\
 $\implies \min\limits_{[2;19]}f(x) = -32\sqrt{2}$ at $x=2\sqrt{2} $ \\
 \textbf{Sentence 3.}$\mathbb{D} = \R$ $f(x)$ continuous on the interval $[\frac{1}{2};2]$ \\
$y'=2x-\frac{2}{x^2}$ \\
$y'=0 \iff $
\begin{math}
	\left[ \begin{array}{l}
			x = 1 \in [\frac{1}{2};2]\\
	    x = 0 \not\in [\frac{1}{2};2]
	\end{array} \right. 
\end{math}\\
$y(\frac{1}{2})=\frac{17}{4}$ \\
$y(2)=5$\\ 
$y(1)=3$\\
$\implies \min\limits_{[\frac{1}{2};2]}y=3$ at $x=1 \implies m = 3$\\
\textbf{ Sentence 4.}
$\mathbb{D}=\R \implies f(x)$ continuous on the interval $[0;\frac{\pi}{2}]$ \\
$f'(x) = 1 -\sqrt{2}\sin x$\\
$f'(x) = 0 \iff 1 = \sqrt{2}\sin x$\\
$\implies \sin x = \frac{1}{\sqrt{2} } = \frac{\sqrt{2} }{2}(\text{Note: multiply with $\sqrt{2} $)} $ \\
$\iff \sin x = \frac{\sqrt{2} }{2}$ \\
$\iff x = \frac{\pi}{4} \in  [0;\frac{\pi}{2}](Note: \text{we know  $\sin (\frac{\pi}{4}) = \frac{2\sqrt{2} }{2} $})$\\
$f(0) = \sqrt{2} $\\
$f(\frac{\pi}{2})=\frac{\pi}{2}$ \\
$f(\frac{\pi}{4}) =\frac{\pi}{4} + 1 $
$\implies \min\limits_{[0;\frac{\pi}{2}]}f(x)=\sqrt{2}$ at $x=0$\\
\textbf{Sentence 5.}\\
Note: m is a constant
$\mathbb{D} = \R \implies f(x)$ continuous on the interval $[-1;1]$\\
 $y'=x^2-x$\\
  $y'=0 \iff $
  \begin{math}
  	\left[ \begin{array}{l}
			x = 1 \in [-1;1]\\
			x= 0 \in  [-1;1]
  	\end{array} \right. 
  \end{math}\\
  $f(-1) = -\frac{5}{6} + 2m = \frac{1}{6} \implies m = \frac{1}{2} \implies$ with $m =\frac{1}{2}\implies \min\limits_{[-1;1]}y = \frac{1}{6} $\\
  \textbf{Sentence 6.}\\
  $\mathbb{D} = \R \setminus \{3\} \implies$ continuous on the interval $[0;2]$\\
 $y'= -\frac{3}{(x-3)^2} \implies y' < 0,\forall x \in \R /\{3\}$\\
$y(0) = 0 $\\
$y(2) = -2$\\
$\implies \max\limits_{[0;2]}y = 0$ at $x=0$\\
\textbf{Sentence 7.}\\
$\mathbb{D} = \R \implies f(x)$ continuous on the interval $[0;\pi]$\\
$f'(x) = 4(\sin x)^2 \cdot \cos x - \cos x$\\
$= \cos x(4(\sin x)^2 - 1) = 0 \iff$
\begin{math}
	\left[ \begin{array}{l}
	    \cos x = 0 \iff x = \frac{\pi}{2} + k\pi\\
	    4(\sin x)^2 = 0 \iff (\sin x)^2 = \frac{1}{4} \iff \sin x = \pm \frac{1}{2} 
\end{array} \right. 
\end{math} $\iff$
\begin{math}
	\left[ \begin{array}{l}
	    x = \frac{\pi}{2} + k\pi\\
	    x = \frac{\pi}{6} + 2k\pi\\
	    x = \frac{5\pi}{6} + 2k\pi\\
	    x = \frac{7\pi}{6} + 2k\pi \not\in [0;\pi]\\
	    x = \frac{11\pi}{6} + 2k\pi \not\in [0;\pi]
	\end{array} \right. 
\end{math}\\
$f(\frac{\pi}{2}) =\frac{4}{3} $ \\
$f(\frac{\pi}{6})=\frac{2}{3}$ \\
$f(\frac{5\pi}{6})=\frac{2}{3}$ \\
$\implies \frac{4}{3} + \frac{2}{3} = 2$\\
\textbf{Sentence 8.}\\
$\mathbb{D} = \R \implies y$ continuous on the interval $[-4;-1]$\\
$y'=\frac{2x-4}{2\sqrt{x^2-4x}}$\\
$y' = \frac{x-4}{\sqrt{x^2-4x}}$ \\
$y' = 0 \iff \frac{x-4}{\sqrt{x^2-4x} }= 0 \iff$
\begin{math}
	\left[ \begin{array}{l}
	    x - 4 = 0 \\
	    x^2 - 4x \ge 0
	\end{array} \right. 
\end{math} $\iff$
\begin{math}
	\left[ \begin{array}{l}
			x = 4 \not\in  [-1;-4]\\
			x = 0 \not\in  [-1;-4]
	\end{array} \right. 
\end{math}\\
$f(-4)=4\sqrt{2} $ \\
$f(-1)=\sqrt{5} $ \\
$\implies (4\sqrt{2})^2 + (\sqrt{5})^2 = 32 + 5 = 37   $\\
\textbf{Sentence 9.}\\
$\mathbb{D} = \R \implies y$ continuous on the interval $[-4;4]$\\
 $y'=3x^2-9$ \\
 $y'=0 \iff 3x^2-6x-9 = 0 \iff$
 \begin{math}
 	\left[ \begin{array}{l}
 	    x = 3\\
	    x = -1 
 	\end{array} \right. 
 \end{math}\\
 $f(-4)=-41$\\
$ f(4)=15$\\
$f(3)=8$ \\
$f(-1)=40$\\
$\implies M = 40, m = -41 \implies m + M = -41+40 =-1$\\
\textbf{Sentence 10.}\\
$\mathbb{D} = \R \setminus \{1\} \implies y$ continuous on the interval $[2;4]$\\
 $y'=\frac{x^2-2x-3}{(x-1)^2}$ \\
 $y'=0 \iff x^2-2x-3 = 0 \iff$
 \begin{math}
 	\left[ \begin{array}{l}
 	   x =3 \\
	   x =-1 \not\in  [2;4]
 	\end{array} \right. 
 \end{math}\\
 $y(2) =7 $\\
 $y(4) =\frac{19}{3}$ \\
 $y(3)=6$\\
 $\implies \max\limits_{[2;4]}y =7 $ at $x = 2$\\
 \textbf{Sentence 11.}\\
 $\mathbb{D} = \R \implies y$ continuous on the interval $[1;3]$\\
  $y'=3x^2-4x+3$\\
   $y'=0 \iff$ no solution\\
$f(1)=-2$\\
$f(3)=14$\\
 $\implies 14 -(-2) = 16$\\
 \textbf{Sentence 12.}\\
 $\mathbb{D} = \R \implies$ y continuous on the interval $[1;3]$\\
  $y'=3x^2-6x-9$\\
   $y'=0 \iff$ no solution\\
   $f(-4)=-76$\\
    $f(6)=54$ \\
    $\implies 54 -(-76) = 130$\\
\textbf{Sentence 13.Note: Be careful with the absolute value sign}\\
$y = |(x-2)(x-3)| \iff$
\begin{math}
	\left[ \begin{array}{l}
	    y = (x-2)(x-3)\\
	    y = -(x-2)(x-3)
	\end{array} \right. 
\end{math} $\iff$
\begin{math}
	\left[ \begin{array}{l}
	  y=x^2-6x+6\\
	  y = -x^2+6x-6
	\end{array} \right. 
\end{math} $\iff$
\begin{math}
	\left[ \begin{array}{l}
	    y' = 2x - 6\\
	    y'=-2x+6
	\end{array} \right. 
\end{math} $\iff$
\begin{math}
	\left[ \begin{array}{l}
			y' = 0 \iff x = 3 \in  [0;3]\\
			y' = 0 \iff x = x = \in [0;3]
	\end{array} \right. 
\end{math}\\
$y(0) = 6$\\
 $y(3) = 0$\\
 $\implies \max\limits_{[0;3]}y = 6$ at $x = 0$\\
 \textbf{Sentence 14.}\\
 $\mathbb{D} = \R \implies y $ continuous on the interval $[-2;3]$\\
 $y' = 4x^3-8x$\\
  $y'=0 \iff$
  \begin{math}
  	\left[ \begin{array}{l}
			x = 0 \in [-2;3]\\
			x = \pm  \sqrt{2} \in [-2;3]
	\end{array} \right. 
  \end{math}\\

$y(0) = 9$\\
 $y(-2)=9$ \\
 $y(3)=54$\\
  $y(\sqrt{2})=5 $ \\
  $y(-\sqrt{2})=5 $\\
  $\implies \max\limits_{[-2;3]} y = 54$ at $x=3$\\
  \textbf{Sentence 15.}\\
  $\mathbb{D} = \R \implies f(x)$ continuous on the interval $[-3;3]$\\
   $f'(x) = 3x^2 - 3x$\\
    $f'(x) = 0 \iff$
    \begin{math}
    	\left[ \begin{array}{l}
			x = 1 \in  [-3;3]\\
			x = 0 \in [-3;3]
    	\end{array} \right. 
    \end{math}\\
 $f(-3)=-16$\\
  $f(3)=20$\\
   $f(0)=2$ \\
   $f(1)=0$\\
   $\implies \min\limits_{[-3;3]}f(x)=-16$ at $x=-3$\\
   \textbf{Sentence 16.}\\
   $\mathbb{D} = \R \setminus \{-2\}$\\
  $y'= \frac{x^2+4x-5}{(x+2)^2}$ \\
  $y'=0 \iff$
  \begin{math}
  	\left[ \begin{array}{l}
	    x^2+4x-5=0
  	\end{array} \right. 
  \end{math} $\iff$
\begin{math}
	\left[ \begin{array}{l}
			x = 1 \in [0;3]\\
			x = -5 \not\in [0;3]
	\end{array} \right. 
\end{math}\\
$y(0) = \frac{1}{2}$\\
$y(3)=\frac{4}{5}$ \\
$y(1)=0 $\\
$\implies \max\limits_{[0;3]}y =\frac{4}{5}$ at $ x =3 $ \\
\textbf{Sentence 17.}\\
$\mathbb{D} = \R \setminus \{0\} \implies y$ continuous on the interval $[1;4]$\\
$y' = 1  -\frac{9}{x^2}$\\
$y'=0 \iff 1 = \frac{9}{x^2}$\\
$\iff x^2 - 9 = 0 \iff$
\begin{math}
	\left[ \begin{array}{l}
			x = 3  \in [1;4]\\
			x = -3 \not\in [1;4]
	\end{array} \right. 
\end{math}\\
$f(1)=10$\\
$f(4)=\frac{25}{4}$ \\
 $f(3)=6$\\
  $\implies 6 + 10 = 16$\\
  \textbf{Sentence 18.}\\
  a) true beacause : \\
  one print machine = 12 USD  $\implies x $ print machine = $12x$ USD \\
  b) false beacause :\\
  cost 1h = 9 USD $\implies$ cost $x$ hours =  $9 \cdot x$\\
 c)true beacause :\\
 \textbf{ we know :}\\
 workload : 3000\\
 cost $x$ machine :  $12x$\\
Productivity $x$ machine :  $30x$\\
$\implies$ Completion time = workload / productivity $\iff \frac{3000}{30x} = \frac{100}{x}$ \\
$\implies$ Monitoring cost = cost 1 hour * completion time  $\iff 9 \cdot \frac{100}{x} = \frac{900}{x}$\\
$\implies$ Total cost = $\frac{900}{x} + 12x$\\
d)true beacause:\\
Do same as question c), except the workload is 4000 and other problems\\
$\implies \frac{4000}{30x} = \frac{400}{3x}$ \\
$\implies 9 \cdot \frac{400}{3x} = \frac{1200}{x}$ \\
$\implies \frac{1200}{x} + 12x$\\
Let $c(x) = \frac{1200}{x} + 12x$ \\
$c'(x) = -\frac{1200}{x^2} + 12$ \\
$c'(x) = 0 \iff x =  10$\\
we know $x \in [1;14]$\\
$c(1) = 1212$\\
 $c(14)= \frac{1776}{7}$ \\
  $c(10)=240$\\
  $\implies \min\limits_{[1;14]}c(x) = 240$ at $x = 10$\\
  $\implies$ The minium production cost to print all the received publlication is : 240 (USD)
\textbf{Sentence 19.}\\
a) True beacause :\\
30 minute = 0,5 half an hour $\implies c(0,5)=\frac{0,5}{(0,5)^2+1}=0,4$ \\
b) false beacause :\\
$\frac{t}{t^2+1}=0,3$\\
$\iff 0,3t^2 - t + 0,3 = 0 \iff$
\begin{math}
	\left[ \begin{array}{l}
	    t = \frac{1}{3}\\
	    t = 3
	\end{array} \right. 
\end{math}\\
$\implies t_{min} = \frac{1}{3}$ \\
c) true beacause:\\
$t > 0 \implies t \in (0;+\infty)$\\
$c'(t)=\frac{1-t^2}{(t^2+1)^2}$ \\
$c'(t)=0 \iff$
\begin{math}
	\left[ \begin{array}{l}
			t = -1 \not\in (0;+\infty)\\
			t = 1 \in (0;+\infty)
	\end{array} \right. 
\end{math}\\
\begin{figure}[ht]
    \centering
    \incfig{figuare}
    \caption{figuare}
    \label{fig:figuare}
\end{figure}\\
$\implies \max\limits_{[0;+\infty]}c(t) = 0,5$ at $t = 1$\\
d) true beacause 
with $t= 1 \implies c(1) = 0,5$\\
\textbf{Sentence 20:}\\
\textbf{Note: You can use a Casio calculator to test each answer.}\\
\textbf{Solve:}\\
constraint: x > 5 $\implies x \in (5;+\infty)$\\
$c'(x) = 2 - \frac{2}{(x-6)^2}$ \\
$c'(x) = 0 \iff$
\begin{math}
	\left[ \begin{array}{l}
	    x = 5 \not\in (5;+\infty)\\
	    x = 7 \in (5;+\infty)
	\end{array} \right. 
\end{math}\\
$\implies c(7) = 20$\\
$\implies \min\limits_{(5;+\infty)}c(x) = 20$ at $x=7$\\
\textbf{Sentence 21:}\\
Let SB = $x \implies SA = 4-x ( 0 < x < 4) $\\
$SC = \sqrt{1+x^2}$(Hypotenuse of triangle SBC)\\
We know : $AC = SC + SA$\\ 
Assume that:  $SA$ goes underground\\
Let function cost is  $C(x)$\\
We have:\\
 $C(x) = 5000\sqrt{1+x^2}+3000(4-x) $ \\
 $C'(x) = \frac{5000x}{\sqrt{1+x^2} } - 3000$ \\
 $C'(x) = 0 \iff$
 \begin{math}
 	\left[ \begin{array}{l}
 	    x = \frac{3}{4} \in (0;4)\\
	    x = -\frac{3}{4} \not\in (0;4)
 	\end{array} \right. 
 \end{math}\\
Variation table:\\

\begin{figure}[H]
    \centering
    \incfig{varvar}
    \caption{varvar}
    \label{fig:varvar}
\end{figure}
$\implies \min\limits_{(0;4)} C(x) = 16000$ at  $x = \frac{3}{4}$ \\
$\implies SA = 4 - \frac{3}{4} = \frac{13}{4}$ \\
\textbf{Sentence 22:}\\
Sumize:\\
give : 180m materials\\
\textbf{Method 1: Using AM-GM inequality}\\
Let $x$ is width of rectangle\\
let  $y$ is length of rectangle\\
 $\implies x + 2y = 180$\\
 $\implies x = 180 -2y$\\
$\implies S = (180-2y)\cdot y = \frac{1}{2}\cdot 2y\cdot (180-2y) \le \frac{1}{2} \cdot \frac{(2y+180-2y)^2}{4} = \frac{180^2}{8} = 4050$ \textbf{[Note:using technique AM-GM inequatity]}\\
$S$ to max when inequality "="  $\implies 2y = 180-2y \implies y = 45$\\
\textbf{Method 2:using derivatives}\\
constraint: $0 < y < 90$\\
Let $S(y)$ is function area\\
 $\implies S(y) = (180-2y)\cdot y$\\
 $\iff S(y) = -2y^2 + 180y$\\
 $S'(y) = -4y + 180$\\
 $S'(y) = 0 \iff y = 45$\\
 Variation table:\\
\begin{figure}[ht]
    \centering
    \incfig{var21}
    \caption{var21}
    \label{fig:var21}
\end{figure}
\\
$\implies \max\limits_{(0;90)}S(y) = 4050$ at $y = 45$\\
\textbf{Sentence 23:}\\
\textbf{Method 1 : Using derivatives}\\
Constraint: $a, x > 0$\\
We know: $V = a^2 \cdot x \implies a = \sqrt{\frac{V}{x}}$ \\
$S = S_p + S_t = 4 \cdot a \cdot h + 2 \cdot a^2 = 4\cdot \sqrt{\frac{V}{x}} \cdot x  + \frac{2V}{x}$\\
$S = f(x) = 4 \cdot \sqrt{Vx} + \frac{2V}{x}$ \textbf{Note: $\frac{1}{\sqrt{x}} \cdot  x = \sqrt{x}, \frac{\sqrt{a} }{\sqrt{b} } = \sqrt{\frac{a}{b}}   $}\\
Consider $f(x) = 4\cdot \sqrt{Vx} + \frac{2V}{x} $ \\
$f'(x) = \frac{2V}{2\sqrt{Vx}} - \frac{2V}{x^2}$ \\
$f'(x) = 0 \iff \frac{2V}{2\sqrt{Vx}} - \frac{2V}{x^2} = 0$ \\
$\iff \frac{2V}{2\sqrt{Vx} } = \frac{2V}{x^2}$ \\
$\iff \frac{V}{\sqrt{Vx} } = \frac{V}{x^2}$ \textbf{Note: $\div$ both sides by 2}\\
$\iff Vx^2 = V\sqrt{Vx} $ \\
$\iff x^2 = \sqrt{Vx} $ \textbf{Note: $\div$ both sides by V}\\
$\iff x^4 = Vx$ \textbf{Note: square both sdes}\\
$\iff x = V^{1/3}$ \textbf{Note: $a^n = b \implies a = b^{1/n}$}\\
+)Sign alnalys tips (chapter 1 - lecture 1):\\
\begin{figure}[H]
    \centering
    \incfig{trick-table}
    \caption{trick table}
    \label{fig:trick-table}
\end{figure}
\textbf{Method 2: Using the AM-GM inequality}\\
We have: \\
$4 \cdot  \sqrt{Vx} + \frac{2V}{x} = \frac{2V}{x} + 2\sqrt{Vx} + 2\sqrt{Vx}  \ge 6 \sqrt[3]{V^2} $\\
The equality $"="$ holds when $\frac{2V}{x} = 2\sqrt{Vx} \implies x = \sqrt[3]{V}$\\
\textbf{Note:}\\
1. As b = c, it suffices to compare a with b.)\\
2. $\sqrt{x} = x^{\frac{1}{2}} $ \\
\textbf{Sentence 24:}\\
\textbf{Method 1: Using the deratives}\\
$\to $ Make a rectangular gift box ( with a lid ) $\implies$ need to calculate $S = S_p + S_t$\\
  $\to V = 200(cm^3) \iff 0,0002(m^3)$\\
$\to h = 2(cm) \iff h = 0,02(m)$\\
$\to l = x(cm) \iff \frac{x}{100}(m) (x>0)$ \\
$\to V = w \cdot h \cdot l \implies w = \frac{V}{l\cdot h} = \frac{0,0002}{\frac{x}{100} \cdot 0,02} = \frac{1}{x}$\\
$\to S = S_p + S_t = 2(lw+lh+wh)$\\
$\to $ Let $S=f(x):$\\
We consider  $f(x) = 2(\frac{x}{100} \cdot \frac{1}{x} + \frac{x}{100} \cdot 0,02 + \frac{1}{x} \cdot 0,02)$ \\
$= 0,02 + 0,0004x + \frac{0,04}{x}$ \\
$f'(x) = 0,0004- \frac{0,04}{x^2}$\\
$f'(x) = 0 \iff$
\begin{math}
	\left[ \begin{array}{l}
	    x = 10 \in (0; +\infty)\\
	    x = -10 \not\in (0; +\infty)
	\end{array} \right. 
\end{math}\\

\begin{figure}[H]
    \centering
    \incfig{variation-table}
    \caption{VARIATION TABLE}
    \label{fig:variation-table}
\end{figure}
\textbf{Method 2: Using the AM-GM inequality}\\
We have: $A = 0,02 + 0,0004x + \frac{0,04}{x}$ \\
(Do not select the const) $\implies 0,0004x + \frac{0,04}{x} \ge 2\sqrt{0,0004x \cdot \frac{0,04}{x}} = 0,008 $ \\
Thus: $A = 0,02 + 0,0004x + \frac{0,04}{x} \ge 0,02 + 0,008 = 0,028$\\
The equality "=" holds when : $0,0004x = \frac{0,04}{x} \implies x = 10$ (beacause $x>0$)




 \end{document}


