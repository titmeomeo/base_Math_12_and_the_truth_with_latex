\documentclass[12pt, a4paper]{article}
\usepackage{float}
\usepackage{graphicx}
\usepackage{amsmath, amssymb}
\newcommand{\R}{\mathbb{R}}
% Nhúng cấu trúc chung
\usepackage[utf8]{inputenc}
\usepackage[T5]{fontenc} % Hỗ trợ tiếng Việt
\usepackage{amsmath}
\usepackage{amssymb}
\usepackage[a4paper, margin=1in]{geometry}

% --- CÁC GÓI BẮT BUỘC CHO VIỆC CHÈN HÌNH (PHẦN BỊ THIẾU) ---
\usepackage{graphicx}      % Gói này anh đã có
\usepackage{import}
\usepackage{pdfpages}
\usepackage{transparent}

% --- ĐỊNH NGHĨA LỆNH \incfig (PHẦN BỊ THIẾU) ---
\newcommand{\incfig}[1]{%
    \def\svgwidth{\columnwidth}
    \import{./figures/}{#1.pdf_tex}
}

% --- ĐỊNH NGHĨA LỆNH \lecture (Phần này của anh đã đúng) ---
\usepackage{xifthen}

\makeatletter
\def\@lecture{}%
\newcommand{\lecture}[3]{
    \ifthenelse{\isempty{#3}}{%
        \def\@lecture{Lecture #1}%
    }{%
        \def\@lecture{Lecture #1: #3}%
    }%
    \subsection*{\@lecture}
    \marginpar{\small\textsf{\mbox{#2}}}
}
\makeatother
% --- KẾT THÚC ĐỊNH NGHĨA LỆNH \lecture ---


% Nếu \lecture chưa được định nghĩa, định nghĩa tạm
\providecommand{\lecture}[3]{%
  \section*{Lecture #1: #3 (#2)}
}
\geometry{
 a4paper,
 total={170mm,257mm},
 left=20mm,
 top=20mm,
}
\begin{document}

\lecture{5}{Fri 12 Sep 2025 6:35}{Lecture 5}
homework

1. \\

C. $\implies \text{true}$\\
Beacause: in range $(-3, -2)$ have Max = -4, $x = 0$\\
B $\implies$ beacause: 16 not in range $(-3,-2)$\\

2. Can have multiple maxima: $\pm 1$ are maxima $\implies -1$ is MAX valued $\implies$ A true \\

3.\\
A true $\implies$ because $y = f(x)$ is increasing function above $(-\infty, 0]$ and $f(x)$ $\implies$ range of $x$ is $(-\infty, 0]$ $\implies$ when $x$ max ($x = 0$) $\implies f(0) = 1$ $\implies f(x) \le f(0) = 1$ and $f(x) = 1$ $\forall x \in (-\infty, 0]$\\

6.\\
Distinguish between:\\
+ local maximum / local minimum: in a neighborhood, local\\
+ global maximum / global minimum: overall, global (all range)\\
C is incorrect because as $y \to \pm \infty$, there is no greatest or least value; there are only two local extrema: a local minimum at -2 and a local maximum at 2. Therefore, A is correct.\\

9.\\
$y = x \ln(x)$\\
$y' = \ln(x) + x \cdot \dfrac{1}{x}$\\
$y' = 0 \iff \ln(x) + 1 = 0$\\
$\iff \ln(x) = -1$\\
$\iff x = e^{-1} = \frac{1}{e}$\\

Variation table:\\
\begin{figure}[h]
    \centering
    \incfig{variation-table-1}
    \caption{Variation table 1}
    \label{fig:variation-table-1}
\end{figure}

Min $y = -\frac{1}{e}$ when $x = \frac{1}{e}$, $x \in (0, e)$\\

11.\\
$D = \R$\\
$f(x) = \sin^4x-2 \cdot (1-\sin^2x)+1 [\sin^2x+\cos^2x=1]$\\
$\iff \sin^4x-2+2\sin^2x+1$\\
Since $t = \sin^2 x \ge 0$ and $\sin^2 x \le 1$, we have $t \in [0,1]$.\\
Let $\sin^2x = t \in [0,1]$.\\
$g(t)=t^2+2t-1$\\
 $g'(t)=2t+2=0$\\
 $\iff t=-1$\\
 $f(0)=-1$\\
  $f(1)=2$\\
   $\implies 2-1=1$\\
12.\\
$D=[-2;2]$\\
 $y'=\sqrt{4-x^2}-\frac{2x(x+2)}{2\sqrt{4-x^2}} $ \\
 $\iff y'=\sqrt{4-x^2}-\frac{x(x+2)}{\sqrt{4-x^2}} $ \\
 $\iff 4-x^2=x^2+2x$\\
 $\iff 2x^2+2x-4=0$ $\iff$
\begin{math}
	\left[ \begin{array}{l}
	    x = 1\\
	    x = -2
	\end{array} \right. 
\end{math}\\ 
$f(-2)=0$\\
 $f(1)=3\sqrt{3} $ \\
 $f(2)=0$\\

 14.\\
 $D =[-2;2]$\\
 $y'=3-\frac{4x}{2\sqrt{4-x^2}}$ \\
 $\iff 3-\frac{2x}{\sqrt{4-x^2}} = 0$ \\
 $\iff 3=\frac{2x}{\sqrt{4-x^2}}$ \\
 \begin{math}
 	\iff 3\sqrt{4-x^2}=2x\\
	\iff \sqrt{4-x^2}=\frac{2x}{3}\\
	\iff 4-x^2=\frac{4x^2}{9}\\
	\iff 36-9x^2=4x^2\\
	\iff 36-13x^2=0 \iff
	\left[ \begin{array}{l}
	    x = \frac{6\sqrt{13}}{13}\\
	    x=-\frac{6\sqrt{13} }{13}
	\end{array} \right. 
 \end{math}\\
\begin{figure}[ht]
    \centering
    \incfig{variable-table-of-ex14}
    \caption{Variable table of ex14}
    \label{fig:variable-table-of-ex14}
\end{figure} \\

\begin{math}
	\implies a = 13\\
	\implies b = \frac{6\sqrt{13} }{13} \iff \frac{6}{13} \cdot \sqrt{13}\\
	\\
	\iff \frac{6}{13} \cdot \frac{13}{\sqrt{13} }, \text{Note:} (\sqrt{x} = \frac{x}{\sqrt{x} }), (13:13=1)  \\
	\iff \frac{6}{\sqrt{13} }\\
	\implies b = 6\\
	\implies  |a+b| = 13+6=19
\end{math}\\

21.\\
 \begin{math}
	 D = \R / \{1\} \\
	 y'=\frac{3}{(x+1)^2}\\
	 y'=0 \iff \frac{3}{(x+1)^2} > 0, \forall x \in \R
\end{math}

\begin{figure}[ht]
    \centering
    \incfig{variable-of-ex-21}
    \caption{Variable of ex 21}
    \label{fig:variable-of-ex-21}
\end{figure}
-> a true \\
\begin{figure}[H]
  \centering
  \incfig{part-b}
  \caption{part b}
  \label{fig:part-b}
\end{figure}

-> b false \\
$\implies$ c. true \\
d. False because $y$ is a linear function that is not continuous at $x = -1$, and $(a;b)$ is an open interval.\\
22.\\
a) \[
\max\limits_{[-3;2]} f(x)=f(2)=6
\]
$\implies$ false\\
b) $6-(-5)=11$\\
c) Have  $-\infty$ should don't have min in range $[1;+\infty)$\\
d)\\
Let $t = 2\sin x-1$\\
As we know range of $\sin x$ is $[-1;1]$\\
 $\implies -1 \le \sin x \le 1$\\
 $\iff -2 \le 2\sin x \le 2$(multiply by 2)\\
 $-3 \le 2\sin x-1 \le 1$ (subtract 1)\\
 $\implies t \in [-3;1]$\\
 $\implies$ Max in range $[-3;1]$ is f(0) = 3 \\
 $\implies$ true.\\
 
 23.\\
 b) $\exists x \in  +\infty$ the function $\implies$ has no maximum on $\R$\\
 c) 5 + 3 = 8 $\implies$ false \\
 d)\\
 $g(x) = f(4x-x^2) + \frac{1}{3}x^3 -3x^2+8x+\frac{1}{3}$ \\
 $g'(x) = (4-2x) \cdot f'(4x-x^2) + x^2 - 6x + 8$\\
        = $2(2 -x) \cdot f'(4x-x^2) + (x-2)(x-4)$\\
	= $2(2-x) \cdot f'(4x-x^2) -(2 -x)(x-4)$ <*>\\ 
	= $(2-x)\cdot 2 \cdot f'(4x-x^2)-x+4$\\
	= $(2-x) \cdot 2 \cdot  f'(4x-x^2) +4-x$\\
	\text{Note: <*> is the crucial transformation step, changing the sign of } $(x-2)$ \text{ to } $-(2-x)$.

	 With $x \in [1;3] \implies 4-x \iff 4 - 3 = 1 \implies 4 - x > 0$\\
	Let $h(x) = (4x-x^2) \iff h'(x) = 4 - 2x \iff x = 2$\\
	Change the limits from $x$ to $u$.\\
	$\implies h(1) = 3$\\
	$\implies h(2) = 4$\\
	$\implies h(3) = 3$\\
	Thus with $x \in [1;3] \implies h(x) \in [3;4]$ should $f'(4x-x^2) \ge 0$\\
	$\implies 2f'(4x-x^2) + 4 - x \ge 0,  \forall x \in [1;3]$\\
	We get $g'(x) = 0 \iff 2 - x = 0 \iff x = 2$\\
	The variation table it as follows:\\
\begin{figure}[ht]
    \centering
    \incfig{variation-table-of-ex-23}
    \caption{variation table of ex 23}
    \label{fig:variation-table-of-ex-23}
\end{figure}\\
$\implies \max\limits_{[1;3]} g(x) = g(2) = f(4) + 7 = 12$\\
$\implies$ true\\
24.\\
a)From the graph $f'(x)$, we obtain the variation table of $f(x)$ on the interval  $[0;5]$\\


\begin{figure}[H]
    \centering
    \incfig{ex_24}
    \caption{ex_24}
    \label{fig:ex_24}
\end{figure}

Obtain: $\max\limits_{[0;5]}f(x) = f(3) \implies$ proposition false\\
b)Beacause function $f$ decreasing on $[4;5]$, we have $\max\limits_{[4;5]} f(x)$ is $f(4)$\\
$\implies x_0 = 4$\\
$\iff 2 \cdot 4^2 + 4 = 36 \implies$ proposition true.\\
c)We get $f(x) \ge f(0),\forall x \in [0;1] \implies -f(x) \le -f(0), \forall x \in [0;1]$ and $-f(x)=-f(0) \iff x = 0 \implies \max\limits_{[0;1]}[-f(x)] = -f(0) \implies$ proposition false.\\
d)We have :\\
$f(0) + f(1) -2f(3) = f(5) - f(4)$\\
$ \iff f(0) - f(5) = -f(4) - f(1) + 2f(3)$\\
we get $f(3) > f(1) \implies 2f(3) -f(1) > 0, \forall x \in [0;5]$\\
we get $f(4) < 0 \implies -f(4) > 0 \implies 2f(3) - f(4) > 0, \forall x \in [0;5]$\\
$\implies f(0) - f(5) > 0$\\
$\iff f(0) > f(5)$\\
$\implies \min\limits_{[0;5]}f(x) = f(5)$\\
$\implies$ Proposition true.


 





\end{document}

