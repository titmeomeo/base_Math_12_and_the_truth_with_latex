% --- PHẦN 1: KHAI BÁO KHUNG TÀI LIỆU ---



\documentclass[12pt]{article} % Chọn loại tài liệu là "article" với cỡ chữ 12



% --- PHẦN 2: KHAI BÁO CÁC GÓI MỞ RỘNG ---
\usepackage{amsmath} % For math environments
\usepackage{enumitem}  % For customized lists


% Hỗ trợ gõ trực tiếp Tiếng Việt (UTF-8)

\usepackage[utf8]{inputenc}



% Hỗ trợ hiển thị và xuống dòng đúng cho Tiếng Việt

\usepackage[T5]{fontenc}

\usepackage[vietnamese]{babel}



% Gói để tùy chỉnh lề trang (tùy chọn)

\usepackage[a4paper, margin=1in]{geometry}



% --- PHẦN 3: ĐỊNH NGHĨA LỆNH TÙY CHỈNH ---

% Ghi chú: Lệnh \lecture của bạn đã được định nghĩa trong template, 

% đây chỉ là một ví dụ để file có thể chạy được.

\newcommand{\lecture}[3]{

    \section*{Lecture #1} % In ra "Bài giảng số 7"

    \large\textit{#2} % In ra ngày tháng

    \par\vspace{1em}\hrule % Thêm một đường kẻ ngang

}





% --- PHẦN 4: BẮT ĐẦU NỘI DUNG TÀI LIỆU ---



\begin{document}



% Bây giờ lệnh \lecture đã được định nghĩa và có thể sử dụng

\lecture{7}{Thu 02 Oct 2025 08:04}{Lecture bonus AM-GM inequality}

\section*{AM-GM Min/Max Cheat Sheet}
\hrulefill % Adds a horizontal line

\begin{enumerate}[label=\textbf{\arabic*.}]
    \item \textbf{Recognition}
    \begin{itemize}[leftmargin=*]
        \item \textbf{When:} Sum of positive terms with inverse powers (e.g., $x^a, 1/x^b$).
        \item \textbf{Action:} Use AM-GM.
    \end{itemize}
    
    \item \textbf{Balancing Exponents}
    \begin{itemize}[leftmargin=*]
        \item \textbf{Goal:} Make the product a \textbf{constant}.
        \item \textbf{Action:} Split terms so the sum of exponents $= 0$.
    \end{itemize}
    
    \item \textbf{Term Creation (Add \& Subtract)}
    \begin{itemize}[leftmargin=*]
        \item \textbf{Problem:} Terms don't cancel (e.g., $x$ and $\frac{1}{x-a}$).
        \item \textbf{Action:} Add \& subtract to create matching terms (e.g., create $(x-a)$).
    \end{itemize}
    
    \item \textbf{Point of Extremum}
    \begin{itemize}[leftmargin=*]
        \item \textbf{Problem:} Direct method fails or there is a boundary condition (e.g., $x \ge k$).
        \item \textbf{Action:} \textbf{Guess} the min point ($x=k$), then split terms to \textbf{force equality} ($a=b$) at that point.
    \end{itemize}
\end{enumerate}
\section*{Giới thiệu}
Đây là tài liệu tổng hợp các kỹ thuật, mẹo và chiến lược quan trọng để sử dụng bất đẳng thức AM-GM (Cô-si) một cách tự tin và chính xác, đặc biệt trong các bài toán tìm giá trị nhỏ nhất (min) và lớn nhất (max).

\section{Dấu Hiệu Nhận Biết "Bài Toán Tủ"}
Hãy tìm kiếm các bài toán có đủ những đặc điểm sau:
\begin{itemize}
    \item \textbf{Mục tiêu:} Tìm giá trị lớn nhất (GTLN) hoặc giá trị nhỏ nhất (GTNN).
    \item \textbf{Miền xác định:} Các biến số đều là số dương (ví dụ: $x > 0$).
    \item \textbf{Dạng biểu thức:} Thường là tổng của các số hạng.
    \item \textbf{Dấu hiệu vàng:} Có sự xuất hiện của biến số ở cả \textbf{tử số} và \textbf{mẫu số}, tạo khả năng triệt tiêu lẫn nhau (ví dụ: $x^a$ và $\frac{1}{x^b}$).
\end{itemize}

\section{Tip 1: Kỹ Thuật Thêm Bớt Hằng Số}
\textbf{Nguyên tắc:} Hằng số không tham gia vào việc cân bằng "điểm rơi". Hãy tách riêng nó ra.

\textbf{Ví dụ:} Tìm GTNN của $P = x^2 + \frac{2}{x} + 5$ với $x > 0$.
\begin{enumerate}
    \item \textbf{Tách riêng hằng số:} Xét phần chứa biến $A = x^2 + \frac{2}{x}$.
    \item \textbf{Tìm GTNN của phần biến:} Áp dụng AM-GM cho 3 số ($x^2, \frac{1}{x}, \frac{1}{x}$), ta có $A_{min} = 3$ tại $x=1$.
    \item \textbf{Cộng lại hằng số:} Vậy $P_{min} = A_{min} + 5 = 3 + 5 = 8$, cũng đạt được tại $x=1$.
\end{enumerate}

\section{Tip 2: Kỹ Thuật "Ép" Điểm Rơi (Nâng cao)}
\textbf{Nguyên tắc:} Khi điều kiện bài toán ép GTNN/GTLN xảy ra tại một điểm $x_0$ không phải là "điểm rơi tự nhiên", ta phải tách các số hạng để dấu "=" của AM-GM xảy ra đúng tại $x_0$.

\textbf{Ví dụ:} Tìm GTNN của $P = x + \frac{1}{x}$ với điều kiện $x \ge 2$.
\begin{enumerate}
    \item \textbf{Dự đoán điểm rơi:} GTNN sẽ xảy ra tại biên, tức là $x_0 = 2$.
    \item \textbf{Tìm hệ số tách:} Ta cần tách $x$ thành $(ax + \dots)$ để ghép cặp với $\frac{1}{x}$. Dấu "=" phải xảy ra tại $x=2$:
    $$ ax = \frac{1}{x} \implies a = \frac{1}{x^2} = \frac{1}{2^2} = \frac{1}{4} $$
    Vậy ta cần tách ra một lượng là $\frac{x}{4}$.
    \item \textbf{Viết lại và áp dụng AM-GM:}
    \begin{align*}
        P &= \left( \frac{x}{4} + \frac{1}{x} \right) + \frac{3x}{4} \\
        % Áp dụng AM-GM cho phần trong ngoặc
        &\ge 2\sqrt{\frac{x}{4} \cdot \frac{1}{x}} + \frac{3x}{4} \\
        &\ge 2\sqrt{\frac{1}{4}} + \frac{3x}{4} \\
        &\ge 1 + \frac{3x}{4}
    \end{align*}
    \item \textbf{Đánh giá phần còn lại:} Vì $x \ge 2$, ta có $\frac{3x}{4} \ge \frac{3 \cdot 2}{4} = \frac{3}{2}$.
    \item \textbf{Kết luận:}
    $$ P \ge 1 + \frac{3}{2} = \frac{5}{2} $$
    Vậy $P_{min} = \frac{5}{2}$, đạt được khi $x=2$.
\end{enumerate}

\section{Tip 3: Luôn Kiểm Tra Lại Bằng Đạo Hàm}
\textbf{Nguyên tắc:} Nếu đã học, đạo hàm là công cụ mạnh nhất và đáng tin cậy nhất để xác minh kết quả. Đây là "lưới an toàn" của bạn.

\textbf{Ví dụ:} Kiểm tra lại GTNN của $P = x^2 + \frac{2}{x}$.
\begin{enumerate}
    \item \textbf{Tính đạo hàm:} $P' = 2x - \frac{2}{x^2}$.
    \item \textbf{Tìm điểm dừng:}
    $$ P' = 0 \iff 2x = \frac{2}{x^2} \iff 2x^3 = 2 \iff x^3 = 1 \iff x = 1 $$
    \item \textbf{Kết quả:} Cực trị xảy ra tại $x=1$, hoàn toàn trùng khớp với kết quả từ AM-GM.
\end{enumerate}
\section*{Ví dụ bổ sung: Dùng AM-GM theo kiểu “bập bênh”}

\textbf{Đề bài:} Tìm giá trị nhỏ nhất của biểu thức:
\[
P = x^2 + \frac{2}{x} \quad \text{với } x > 0
\]

Giá trị nhỏ nhất của biểu thức là $3$, đạt được tại $x = 1$.

Để giải bài này, ta tiếp tục dùng "mẹo bập bênh" để áp dụng bất đẳng thức AM-GM (Cô-si).

\subsection*{Phân Tích và Áp Dụng AM-GM}
\begin{enumerate}
    \item \textbf{Xác định "trọng số" (bậc lũy thừa):}  
    Một bên là $x^2$ có bậc là $2$.  
    Bên kia là $\frac{2}{x}$ có $x$ ở dưới mẫu với bậc là $1$.

    \item \textbf{Cân bằng "bập bênh":}  
    Ta cần cân bằng: $2 \times m = 1 \times n$.  
    Tráo đổi hệ số, ta chọn cặp đơn giản nhất là $m = 1$ và $n = 2$.  
    \textbf{Kết luận:} Ta cần $1$ phần từ $x^2$ và $2$ phần từ $\frac{2}{x}$.

    \item \textbf{"Chia bánh" và áp dụng AM-GM:}  
    $1$ phần từ $x^2$ chính là $x^2$,  
    $2$ phần từ $\frac{2}{x}$ tức là ta chia $\frac{2}{x}$ thành 2 phần bằng nhau:  
    \[
    \frac{2}{x} = \frac{2}{2x} + \frac{2}{2x}
    \]
    \textit{(Ghi chú: có thể tách thành $\frac{1}{x} + \frac{1}{x}$ cho tiện, kết quả không đổi.)}
\end{enumerate}

Bây giờ ta có 3 số hạng để áp dụng AM-GM là $x^2$, $\frac{2}{2x}$ và $\frac{2}{2x}$:
\[
\begin{align*}
P = x^2 + \frac{2}{x} &= x^2 + \frac{2}{2x} + \frac{2}{2x} \\
&\ge 3 \cdot \sqrt[3]{x^2 \cdot \frac{2}{2x} \cdot \frac{2}{2x}} \\
&= 3 \cdot \sqrt[3]{x^2 \cdot \frac{4}{4x^2}} \\
&= 3 \cdot \sqrt[3]{1} = 3
\end{align*}
\]

\subsection*{Tìm Điểm Rơi (Dấu "=" Xảy Ra)}
Giá trị nhỏ nhất đạt được khi các số hạng bằng nhau:
\[
\begin{cases}
x^2 = \frac{2}{2x} \\
\Rightarrow x^2 = \frac{1}{x} \\
\Rightarrow x^3 = 1 \Rightarrow x = 1
\end{cases}
\]
Vì $x=1$ thuộc miền xác định $(0; +\infty)$, nên:
\[
\boxed{\text{Giá trị nhỏ nhất của biểu thức là } 3 \text{ tại } x = 1}
\]


\end{document}
