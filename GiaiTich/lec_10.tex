\documentclass[12pt, a4paper]{article}
\usepackage[utf8]{inputenc}
\usepackage[T5]{fontenc} 
\usepackage{amsmath, amssymb}
\usepackage{pdfpages}
\usepackage{graphicx}
\usepackage{float}

\newcommand{\R}{\mathbb{R}}
\newcommand{\lecture}[3]{%
  \section*{Lecture #1: #3 (#2)}
}
\begin{document}

\lecture{8}{Mon 20 Oct 2025 18:58}{Lecture 8} 

\section*{The asymptote of the graph of a function – Part 1}
\textbf{HOMEWORK LECTURE 08:} \par
\textbf{Sentence 03:}\par
$\lim\limits_{x \to +\infty}\frac{3x-2}{4-x}=\lim\limits_{x \to +\infty}=-3$ \par
$\lim\limits_{x \to -\infty}\frac{3x-2}{4-x}=\lim\limits_{x \to -\infty}-3$ \par
$\implies y = -3$ is Horizontal Asymptote of function\par 
\textbf{Sentence 04:}\par
Domain : $2x-1 = 0 \implies x = \frac{1}{2}$ \par
$\lim\limits_{x \to \frac{1}{2}^+}\frac{2x+1}{2x-1}=+\infty $ \par
$\implies x = \frac{1}{2}$ is the vertical asymptote of the graph of the function \par
 \textbf{Sentence 7:}\par
Domain $x^2-3x+2 = 0 \iff$
 \begin{math}
 	\left[ \begin{array}{l}
 	    x = 2\\
	    x = 1
 	\end{array} \right. 
 \end{math}\par
$\lim\limits_{x \to 2+}\frac{\sqrt{x+3}-2 }{x^2-3x+2}=+\infty \implies x = 2$ is the vertical asymptote of the graph of the function\par
$\lim\limits_{x \to 1^+}\frac{\sqrt{x+3}-2 }{x^2-3x+2}=\lim\limits_{x \to 1^+}\frac{\left( \sqrt{x+3}-2  \right)\left( \sqrt{x+3}+2  \right)  }{x^2-3x+2\left( \sqrt{x+3}+2  \right)}=\lim\limits_{x \to 1^+}\frac{x-1}{\left( x-2 \right)\left( x-1 \right)\left( \sqrt{x+3}+2\right)}=\lim\limits_{x \to 1^+}\frac{1}{\left( x-2 \right)\left( \sqrt{x+3}+2\right)}=-\frac{1}{4} \implies$ not have vertical asymptote \par
\textbf{Sentence 8:}\par
We have : $\lim\limits_{x \to \infty}f(x) = 3$\par
$\lim\limits_{x \to \infty}-3f(x)+11 \iff -3\cdot \lim\limits_{x \to \infty}f(x) + 11 = -3\cdot 3+11 = 2 \implies y = 2$\par
\textbf{Sentence 9:}\par
Domain $x^2+x=0 \iff $
\begin{math}
	\left[ \begin{array}{l}
	    x = -1\\
	    x = 0
	\end{array} \right. 
\end{math} \par
$\lim\limits_{x \to -1^+}\frac{\sqrt{x+9}-3 }{x^2+x}= \lim\limits_{x \to -1^+}\frac{\sqrt{x+9}-3 }{x\left( x+1 \right) }=+\infty \implies x = -1$ is the veritical asymptote of the graph of the function.\par
$\lim\limits_{x \to 0^+}\frac{\sqrt{x+9}-3 }{x^2+x}=\lim\limits_{x \to 0^+}\frac{\left( \sqrt{x+9}-3  \right)\left( \sqrt{x+9}+3  \right)  }{x^2+x\left( \sqrt{x+9}+3\right) }= \lim\limits_{x \to 0^+}\frac{x}{x\left(x+1)\right \left( \sqrt{x+9} +3 \right)  \right) }=\lim\limits_{x \to 0^+}\frac{1}{\left( x+1 \right)\left( \sqrt{x+9}+3  \right)  }= \frac{1}{6} \implies$ not  have the veritical asymptote of the graph of the function.






\end{document}
